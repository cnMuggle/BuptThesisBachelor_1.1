\documentclass[a4paper,cs4size]{ctexrep}
\usepackage{float}
\usepackage[top=1in,bottom=1in,left=1in,right=1in]{geometry}
\usepackage{multirow}
\usepackage{tabularx}
%from Internet http://hi.baidu.com/lyunsun/blog/textbf/ad50e354db8537163b293530.html
%实现居中且列宽固定
\newcommand{\PreserveBackslash}[1]{\let\temp=\\#1\let\\=\temp}
\newcolumntype{C}[1]{>{\PreserveBackslash\centering}p{#1}}
\newcolumntype{R}[1]{>{\PreserveBackslash\raggedleft}p{#1}}
\newcolumntype{L}[1]{>{\PreserveBackslash\raggedright}p{#1}}

\usepackage{eso-pic,calc}

\begin{document}
\pagestyle{empty}

\begin{center}
\zihao{-3}\songti{\ziju{1} 北京邮电大学}

{\heiti 本科毕业设计(论文)信息卡}
\end{center}
\noindent
\begin{tabular}{|C{0.2\textwidth}|C{0.25\textwidth}|C{0.06\textwidth}|C{0.12\textwidth}|C{0.11\textwidth}|C{0.1\textwidth}|}
\hline
学院 & 信息与通信工程学院 & 专业 & 通信工程 & 班级 & 06123 \\
\hline
学生姓名 & 樊高峰 & 学号 & 062367 & 班内序号 & 26 \\ 
\hline
指导教师姓名 & 洪波 & 职称 & \multicolumn{3}{c|}{副教授} \\  
\hline 
\end{tabular}
%%%%%%%%%%%%%%%%%画图区%%%%%%%%%%%%%%%%%
\setlength{\unitlength}{1mm}
\noindent\begin{picture}(0,0)
\multiput(0,0)(160,0){2}{\line(0,-1){205}}
\multiput(0,0)(160,0){2}{\line(0,1){17}}
\put(0,-205){\line(1,0){160}}
\end{picture}
%%%%%%%%%%%%%%%%%%%%%%%%%%%%%%%%%%
\noindent 第1-2周记录:

\textbf{3.1-3.7 第一周 任务书,开题报告;}
\begin{enumerate}
\item 完成开题报告和任务书。
\item 对触发器进行了需求分析,对于其在脑-机接口中的应用进行了探讨。设计了原理图,硬件原理图以及软件流程图。对于系统需使用的各个模块进行了探讨,并初步确定设计方案。主芯片选用STM32F103RBT6,蓝牙模块采用重庆金瓯的BMT0704,USB转USART使用CP2102芯片。设计中预留了一路USB通路用于未来扩展实现USB传输Trigger使用。
\item 确定了一个学期的工作计划,每周进度安排。
\item 分析了毕设各个阶段的需求,探讨最后如何验证系统的方案。
\end{enumerate}
\textbf{3.8-3.14 第二周 硬件电路设计;STM32 最小系统,蓝牙模块接口电路,并口电路,USB 数据电路和 USB 电源供电;}
\begin{enumerate}
\item	修改了硬件原理图中的一处错误。确定PCB尺寸为$40mmX70mm$。
\item	根据上周的硬件设计在Protel里画Schematic原理图,对两组USART,一个USB端口,一个8bit并口所需使用的端口位置根据芯片封装LQFP64进行了分配,减少交叉连线。
\item 完成了原理图和最初PCB布局的准备工作,列表归纳了所有使用到的引脚,进行了初步的布局。
\end{enumerate}
\noindent\begin{tabularx}{160mm}{|C{0.2\textwidth}|X|C{0.12\textwidth}|C{0.22\textwidth}|}
\hline
\multirow{2}{*}{指导教师签字} & \hspace*{0.3\textwidth} & \multirow{2}{*}{日期}	& \multirow{2}{*}{2010 年 4 月 26 日} \\ 
 & \hspace*{0.3\textwidth} & \hspace*{0.1\textwidth} & \hspace*{0.1\textwidth} \\ \hline
\end{tabularx}
\noindent 第3-4周记录:

\textbf{3.15-3.21 第三周 硬件电路设计和 PCB 制版提交;}
\begin{enumerate}
\item 进行PCB的布局和布线,修改了PCB的尺寸,从$40mmX70mm$改成\\ \noindent $55mmX95mm$,并确定四个安装孔的位置
\item 修改原理图部分端口设置,使主要通信线路的连线更短。
\item 对于Wireless Trigger无线触发器延迟进行了分析,认识到减少无线传输延迟的方差是关键。同时要控制延迟平均时间在10ms以内。思考了该如何测试延迟时间的方法。 
\end{enumerate}
%%%%%%%%%%%%%%%%%画图区%%%%%%%%%%%%%%%%%
\setlength{\unitlength}{1mm}
\noindent\begin{picture}(0,0)
\multiput(0,0)(160,0){2}{\line(0,-1){57.5}}
\multiput(0,0)(160,0){2}{\line(0,1){29.1}}
\put(0,29.1){\line(1,0){160}}
\end{picture}
%%%%%%%%%%%%%%%%%%%%%%%%%%%%%%%%%%
{\bfseries 3.22-3.28第四周 STM32系统编程,完成基本元操作程序的编写;     }
\begin{enumerate}
\item 用STM32的Firmware library完成了基本的LED亮灭,USART与PC通信,USART使用DMA与PC通信,USART通过DMA使能中断与PC通信,模拟Wireless Trigger最后的数据格式,每次从PC接收10组3bytes的数据后加8pin输入的Trigger返回PC。
\item 为下周的USB通信编程做准备。
\end{enumerate}
\noindent\begin{tabularx}{160mm}{|C{0.2\textwidth}|X|C{0.12\textwidth}|C{0.22\textwidth}|}
\hline
\multirow{2}{*}{指导教师签字} & \hspace*{0.3\textwidth} & \multirow{2}{*}{日期}	& \multirow{2}{*}{2010 年 4 月 26 日} \\ 
 & \hspace*{0.3\textwidth} & \hspace*{0.1\textwidth} & \hspace*{0.1\textwidth} \\ \hline
\end{tabularx}

\vspace*{2ex}
\noindent 注:每2周内容记录在一个表格中,双面打印
\end{document}
